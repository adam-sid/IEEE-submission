\documentclass[conference]{IEEEtran}
\IEEEoverridecommandlockouts
% The preceding line is only needed to identify funding in the first footnote.
%If that is unneeded, please comment it out. Template version as of 6/27/2024

\usepackage{cite}
\usepackage{amsmath,amssymb,amsfonts}
\usepackage{algorithmic}
\usepackage{graphicx}
\usepackage{textcomp}
\usepackage{xcolor}
\usepackage[hyphens]{url}
\usepackage[hidelinks]{hyperref}
\usepackage{breakurl}
\usepackage{tabularx}
\bibliographystyle{IEEEtran}
\def\BibTeX{{\rm B\kern-.05em{\sc i\kern-.025em b}\kern-.08em
    T\kern-.1667em\lower.7ex\hbox{E}\kern-.125emX}}
\begin{document}

%TC:ignore

\title{Forecasting microclimates with a low-cost LoRa enabled weather station
network\\
\thanks{This project is funded by the UK Research Institute}
}

\author{\IEEEauthorblockN{1\textsuperscript{st} Adam Sidnell}
\IEEEauthorblockA{\textit{Computer Science MSc} \\
\textit{University of Bristol}\\
Bristol, United Kingdom \\
adam.sidnell@gmail.com}
\and
\IEEEauthorblockN{2\textsuperscript{nd} Ruzanna Chitchyan}
\IEEEauthorblockA{\textit{Professor in Computer Science} \\
\textit{University of Bristol}\\
Bristol, United Kingdom \\
r.chitchyan@bristol.ac.uk} }

\maketitle

\begin{abstract}
General weather forecasts are often too broad to accurately capture the small
climatic variation that exists on farms. This can lead to sub-optimal decisions
on tasks such as crop spraying, frost prevention or irrigation. While commercial
weather stations exist, there is often a trade-off between their cost and
transmission range. Additionally, these hardware products do not attempt to use
their data to offer farmers improved forecasting on their fields. This paper
presents the prototype design and evaluation for Agriscanner: An end-to-end,
low-cost IoT (Internet of Things) weather station network that incorporates
machine learning (LightGBM) to predict future weather within a field. Data is
streamed to a web application for live visualisation. Evaluation shows the
hardware system achieves a 1,200m range between nodes for an effective range of
2,400m in non-ideal conditions. This was achieved at a cost of £520 for the
complete system (including two sensor nodes, a repeater, and a WiFi gateway)
while outperforming commercial alternatives in range and cost. The machine
learning model showed mixed results, outperforming general forecasts for certain
climate factors (wind-speed) but highlighting the need for a larger, more
diverse training dataset.
\end{abstract}

\begin{IEEEkeywords}
machine learning, smart farming, forecast, microclimate, IoT, LoRa, LightGBM
\end{IEEEkeywords}

%TC:endignore

\section{Introduction}\label{INTRO} 

Accurate weather data is critical for decision making in agriculture. Farmers
rely on forecasts for various daily farming functions with significant financial
and environmental costs attached. For example, it is required by law that
pesticide spraying must be done in low wind speed to prevent spray drift which
can damage nearby ecosystems \cite{defra2006code}. Similarly, spring frosts pose
a significant risk to crop yields; a single frost night can reduce crop yields
by as much as 24\% \cite{drepper2022springfrost}, highlighting the need for an
accurate prediction allowing farmers to take appropriate counter measures (in
the case of frosts this could be installing high power fans over the field).

However, weather forecasts, as provided by the MetOffice in the UK, are based on
macro-scale models which can only give a more general view of weather in a
region and not smaller distinct climate variations that exist within this, known
as microclimates. The models used by weather forecasts are not suitable for
microclimate prediction as they are designed for wide area predictions based on
planetary movements of wind and moisture.

Farmers rank the accuracy and reliability of weather forecasts as one of the
largest limiting factors when making decisions in growing season
\cite{hu2006understanding}. Microclimates can occupy an area of anywhere from
less than a metre across to several hundred metres \cite{wmo2024}, thus multiple
different microclimates can exist within a single field. This need for
hyper-local data was confirmed in preliminary interviews; one farmer noted that
a single orchard, being situated on a large slope, experienced significantly
different wind speeds on one side compared to the other.

To address this issue, this paper presents a complete end-to-end system called
Agriscanner. While commercial IoT weather stations exist, simply viewing current
and historical data is insufficient for effective planning. The primary
contribution of this work is a deployable system that applies an open-source
machine learning procedure using locally collected sensor data to forecast the
microclimate.

This approach is distinct from related local forecasting work in three key
areas. First, unlike studies relying on public datasets or existing sensor
networks, we designed and built the entire low-cost hardware network required to
collect data from remote fields. Second, while other systems focus on complex
deep learning models, we demonstrate the application of an efficient, tree-based
machine learning model (LightGBM) as a more accessible alternative. Third, this
model is applied in a challenging open air environment, as opposed to more
controlled contexts such as greenhouses.


\section{Related Work}\label{REL}

IoT devices have seen widespread adoption in agriculture, with digital solutions
offering the potential to improve yields even in remote areas.
\textcolor{blue}{IoT systems require a network communication layer to allow
either inter-device or device-cloud communication. There are a number of
alternative communication protocols used in IoT architecture, each with distinct
trade-offs. The following section reviews literature on common protocols used in
agriculture (i.e.Bluetooth and Wi-Fi), contrasting them with LoRa which is the
communication protocol used for this project}
\textcolor{red}{Add some related work on non-LoRa IoT, then contrast that to LoRA}

\textcolor{blue}{ \subsection{ Non-LoRa IoT in agriculture} One common
communication standard used in IoT systems is Wi-Fi. In \cite{wifiIoT}, the
authors developed a smart wireless sensor network using 802.11 Wi-Fi modules
(WSN802G) to monitor agricultural parameters such as temperature, air pressure
and soil moisture. Readings from sensors were transmitted to a standard router
and uploaded to the cloud for storage and retrieval. While the system
successfully enabled remote monitoring, Wi-Fi has relatively high power
consumption and while the authors aimed to move to a solar-battery set up as
part of future work future this had not been attempted.}

\textcolor{blue}{Similarly, studies have investigated Bluetooth for agricultural
IoT. In \cite{tacskin2018developing}, the authors explored an IoT sensor network
based on Bluetooth Low Energy (BLE) for greenhouse monitoring. The primary
advantage of this technology is very high power efficiency with the authors
estimating a battery life of over 8 years of continuous transmission. Their
prototype connected ambient light and temperature sensors to a BLE transmitter.
The transmission is then received by a BLE compatible router and data is
uploaded to a web API via WiFi. While this solution offers superior energy
efficiency compared to WiFi, the authors noted that BLE has limited range of
30-100m. This would make make Bluetooth unsuitable for open-field contexts that
lack nearby network connectivity.}

\textcolor{blue}{Research has also examined low-rate wireless personal area
networks (LR-WPAN) for agricultural monitoring. In \cite{xbee}, the authors
developed a Wireless Sensor Network (WSN) using XBee hardware modules operating
within the 802.15.4 communication standard to monitor soil moisture and
temperature. Their system utilized a mesh topology and consisted of sensor
nodes, routers, and a central coordinator to facilitate communication. While the
study demonstrated the low power consumption (62 days) and reliability even when
deployed in open air conditions, the technology's shorter transmission range
necessitates a complex multi-hop network architecture to cover typical farm
fields effectively. }

\textcolor{blue}{These non-LoRa technologies present certain advantages over
LoRa but also come with significant disadvantages in an off-grid context. While
Wi-Fi provides high bandwidth, its high power consumption restricts off-grid
viability. Conversely, while BLE and LR-WPAN offer excellent energy efficiency,
their limited transmission ranges require dense mesh networks using multiple
nodes to to cover typical agricultural areas and be within reach of an internet
gateway connection. In contrast, LoRa addresses these specific constraints by
offering a balance of long-range communication and low power consumption without
the need for complex intermediate infrastructure.}


\textcolor{blue}{\subsection{LoRa IoT in agriculture}}

LoRa (Long Range) is a radio modulation technique that allows for the
transmission of data over very long distances (over 4000 times greater than WiFi
\cite{spiess2019}) while using remarkably little power. This makes it the
preferred technology for the remote, off-grid applications. LoRa receiver can
aslo to distinguish signals even when background noise is ``louder" than the
LoRa signal \cite{vangelista2017}. The main disadvantage of LoRa is a
comparatively low data throughput - around 5 kbps in the configuration used
here. 

Papers examining the effectiveness of LoRa in agricultural applications include
\cite{edgeAiGiaEtAl} where LoRa was used in an edge computing exercise. In this
study the authors used CNN machine learning to create a compressed image that
holds thousands of simulated climate readings. This image can then be sent over
LoRa to a receiver node which can infer the readings of each node from this
single image. While only one sensor node was created for the exercise they also
tested the range of this device at a distance of 200m. This system would be
useful in particularly large networks of LoRa devices where the low data
transfer speed of LoRa would start to be a limiting factor.

The authors in \cite{smartFarmKodaliEtAl} implement a LoRa based weather station
prototype in India. The authors create a node that measures temperature,
humidity and soil moisture in an experimental setting with no field deployment.
Readings are then sent via LoRa to a receiver and can be read manually from the
device's screen or viewed on an IBM dashboard.  

\textcolor{red}{Conclude by saying that LoRa is particularly relevant in fields, with no wifi...}

\textcolor{blue}{I've added some commentary on this to the non-lora section}

\subsection{Weather forecasting microclimates with machine learning}

General weather models operate at magnitudes between 1 and 10 km and
microclimate predictions require models that operate at scales of roughly 100m
or less. Using existing general forecast models for micro-scale predictions is
computationally expensive \cite{blunn2024machine}, and these models have lower
accuracy rates than predictions using machine learning due to the inherent
complexity and non-linear nature of microclimates. Therefore a number of studies
have focused on building bespoke models to predict very local forecasts using
machine learning processes.

A 2021 study by Kumar et al \cite{kumar2021} developed an ML framework called
DeepMC as a part of a Microsoft Research initiative. Their model is able to
predict a variety of climatic variables such as soil moisture, wind speed and
temperature using inputs from weather station forecasts and IoT sensors. They
were able to get up to 90\% accuracy with a 12-120 hour forecast range.

Zanchi et al \cite{zanchi2023harnessing} used physical modelling of local
terrain combined with deep learning (DL) to forecast the microclimate in the
foothills of Lombardy. The objective was to predict the local conditions at the
meter-scale as opposed to the 10km+ scale of regional and global weather
forecasts. The initial model combined data about the morphology of the local
terrain and weather forecast data to provide the input data for two feed-forward
neural networks. These neural networks were trained to predict the local weather
variables using data from 25 sensors deployed in the region being studied.  The
study demonstrated that local predictions were more accurate when using forecast
data from local weather stations as opposed to global climate datasets, but
accuracy was high in both cases. \textcolor{blue}{It was notable that in this
study only 4 of the 25 sensors used ran without failure.}

Blunn et al \cite{blunn2024machine} ran a study focussed on predicting
temperatures in urban environments during heatwaves, using data from eight
heatwaves in London, UK. They used data from the UKV - a high-resolution weather
forecasting model - and from citizen weather stations (CWS). The authors used a
similar model training design to that in this study. A number of ML models were
trained on UKV variables (i.e. a general forecast) and CWS variables (local
sensor data) to bias correct the UKV readings and create a forecast prediction
model that could predict the CWS readings accurately (mean average error:
0.12\(^\circ\)C) compared with the general weather readings from UKV (mean
average error: 0.64\(^\circ\)C). The main points of difference to this paper are
the use of only temperature versus a wider range of variables in this study,
along with the use of custom weather stations here compared to public weather
data.

A recent paper from Abdelmadjid et al 2025 \cite{abdelmadjid2025enhancing} used
online datasets from Kaggle (a public repository of various datasets) to develop
an ML tool to predict changes in temperature and humidity within greenhouses in
response to changes to external weather conditions. They used these data to test
three ML models and three DL models and selected the LightGBM ML model and the
LSTM DL model as the best performing models for prediction. The overall system
design consisted of four LSTM models feeding into the LightGBM model. This
design resulted in 98.45\% accuracy for temperature predictions and 99.61\%
accuracy for humidity predictions. Due to these results LightGBM was also chosen
for this experiment.

\textcolor{red}{Note cold start challenge, and cost of equipment and wifi needs - and conclude that these challenges are not addressed in related work and are addressed in this paper.}

\textcolor{blue}{There are still challenges that these studies do not address.
First, many studies do not provide specific detail on the costs of their sensors
and in some cases rely on high-cost proprietary hardware as in \cite{kumar2021}.
While the authors in \cite{abdelmadjid2025enhancing} give a comprehensive
overview of many ML models in forecast prediction their data come from
pre-existing public datasets, ignoring the practical challenge of deployment.
Studies that do deploy actual hardware such as in \cite{zanchi2023harnessing}
often face reliability issues, in that study the authors had a high failure rate
in their sensors. This paper addresses these gaps by presenting a complete
solution from the hardware to software level. This paper addresses these gaps by
presenting a complete solution from the hardware to the software level. The use
of LoRa allows this solution to work in the open-air environment over greater
distances. This makes the use of ML based forecasting more practical for
producers} \section{Methodology}\label{METH}

\textcolor{red}{methodology is Design Science: Analyse, design, deploy, evaluate improve, repeat. I think we had two cycles, the first resulted in battery doubling and box re-design with shading as well. The 3rd is in progress (deploying at Langford).  Add this as the overview part, then the sub-sections are on specific design choices/solutions.}

\textcolor{blue}{The methodology for this project broadly aligns with the
concept of an iterative design and engineering cycle proposed by
\cite{wieringa2014design}. This involved an analysis of requirements; designing
the software and hardware; deployment of hardware; and finally a qualitative
evaluation. There were two distinct development cycles; in the first cycle
issues with battery power and solar heating were identified that were then
improved upon in the second cycle}

\textcolor{blue}{Presently, a third development cycle is underway. In this
cycle, the weather stations are being modified to include wind direction
measurements and gas detection (carbon dioxide, ammonia, methane and nitrogen
dioxide). On completion, the stations will be deployed to Wyndhurt Farm in South
West England to measure climate conditions and gas emissions from cattle.
However, as this work is ongoing, it remains outside the scope of this paper.}

\subsection{IoT hardware overview}

\begin{figure}[htbp]
\centerline{\includegraphics[width=0.5\textwidth]{figures/network-diagram.png}}
\caption{Network diagram of the system}
\label{fig}
\end{figure}

The design of the hardware consisted of two sensor nodes, a repeater and a
gateway. The purpose for each is outlined below:

\begin{enumerate}
    \item Sensor node: Collects temperature, humidity, wind speed and soil
    moisture data every 6 seconds. These readings are then averaged and sent as
    a single packet each minute to the repeater. Figure \ref{sensor-node} shows
    one of these nodes.
    \item Repeater: Receives LoRa signals from the sensor nodes and then
    immediately re-transmits these to boost range. 
    \item Gateway: A hub that receives LoRa signals and then transmits weather
    data using WiFi.
\end{enumerate}

\begin{figure}[htbp]
\centerline{\includegraphics[width=0.45\textwidth]{figures/node.jpg}}
\caption{Sensor node}
\label{sensor-node}
\end{figure}

All components were commercially available, and assembling the final hardware
required only basic tools. Each device used an RP2040 based microcontroller
(iLabs Challenger LoRa), which provided the necessary computing power and
included built-in LoRa capability for wireless communication. The gateway node
was additionally equipped with a Raspberry Pi to enable WiFi connectivity and
remote access via the VNC Viewer application.

\textcolor{blue}{The microcontrollers were programmed using a distribution of
Python for low-power devices called CircuitPython \cite{circuitpython}. } 

\textcolor{blue}{\subsection{Solar power and battery life}
While developing the nodes, the ability for them to operate without mains power
supply was important to ensure they could be deployed in remote applications.
The electronics were initially tested in a controlled environment on a fully
charged single 2500 mAh 18650 battery. In this configuration, they lasted 19
hours before the battery was entirely drained. However after being tested in the
field over a week, this capacity proved insufficient. During periods of low
sunlight, nodes frequently stopped transmitting at approximately 06:00. To
address this in the second development cycle, a second battery was added in
parallel, doubling the capacity to 5000 mAh. Subsequent monitoring confirmed
this was sufficient to maintain power through the night and recharge fully
during the day.}

\subsection{Weatherproofing}
All sensitive electronics were housed in an IP65-rated waterproof junction box.
Most external components were water resistant by design with the exception of
the soil moisture sensor: In this case the electronics were coated in
non-conductive nail varnish and sealed with heatshrink. All cable entry points
to the main box were sealed with silicone to prevent water ingress.

\textcolor{blue}{\subsection{Solar heating issue}
In the first development cycle, the node was constructed with the temperature
and humidity sensor housed in a small IP55 junction box painted white. However,
data from the initial deployment revealed a significant discrepancy during sunny
conditions. The sensor readings showed a delta of approximately $5^{\circ}C$
compared to local air temperature, indicating that the white paint was not
sufficient to prevent solar radiation affecting the accuracy of readings.}

\textcolor{blue}{To address this in the second development cycle, a custom solar
shield was created using a reshaped aluminium can mounted on a wooden frame.
Both the can and frame were also painted white to improve reflection of solar
energy. This shield was installed to surround the sensor housing, blocking
direct solar radiation while maintaining good airflow over the sensor. After the
installation, readings in sunlight were in line with shaded readings.}

\subsection{Deployment and data collection period}

The nodes were completed and installed in August 2025 in a private garden for a
test deployment collecting local readings from two different locations within
the garden. This helped facilitate easy repairs and updates to the devices.

The system ran continuously in this location from the 15th of August, 2025,
which provided the data used for the machine learning model's training and
evaluation later that month.

\subsection{Webapp design}

Weather data was sent from the gateway to a webapp called Agriscanner. This
webapp allowed for the displaying of current, historic and future (predicted)
weather.

The dashboard (Figure \ref{dashboard}) displayed live weather data from the
nodes with a 1 minute update frequency. Clicking on a particular data point
allows the user to see a graph of past and future weather data in a way that is
familiar to any user of standard forecasting apps (Figure \ref{temperature}).

\begin{figure}[htbp]
\centerline{\includegraphics[width=0.5\textwidth]{figures/main-page.jpg}}
\caption{Webapp main dashboard}
\label{dashboard}
\end{figure}

\begin{figure}[htbp]
\centerline{\includegraphics[width=0.5\textwidth]{figures/temperature-node-one.jpg}}
\caption{Webapp temperature page showing current and predicted weather}
\label{temperature}
\end{figure}

\subsection{Training and deployment of Machine learning algorithm}

To forecast microclimate date up to 48 hours in advance, we trained 10 separate
machine learning models using the LightGBM algorithm. One model was created for
each of the five sensor variables (temperature, humidity, wind speed, gust speed
and soil moisture) for each of the two nodes. LightGBM was selected for its high
performance on tabular climate data as the authors in
\cite{abdelmadjid2025enhancing} show. An additional benefit of this set up is
that LightGBM is compatible with the m2cgen library which allows the conversion
of the final models to individual JavaScript files.

\subsubsection{General forecast source}
The model was trained using both historical sensor readings collected in the
field and general forecast data taken from OpenWeather's \emph{One Call API 3.0}
\cite{openweatherAPI}. 

This weather API was selected as it was free to use and offered a high
resolution of forecast readings, with new current and future weather data
produced every 10 minutes.

\subsubsection{Training procedure}

The final models would take future general weather forecast data as input and
correct for any bias compared to the sensor data. The output was then a machine
learning corrected forecast that was predicted to give a more accurate forecast
of the conditions in the microclimate. This was predicted to improve forecasting
accuracy.

\begin{figure*}[htbp]
\centerline{\includegraphics[width=0.8\textwidth]{figures/machine-learning-diagram2.jpg}}
\caption{Infographic showing training steps for training with LightGBM}
\label{machine-learning}
\end{figure*}

The following steps were followed to train each of the ten models, also shown
visually in Figure
\ref{machine-learning}

\begin{enumerate}
    \item \textbf{Dataset prepared:} A single cleaned dataset was created by
          matching timestamps between the api weather data and the sensor node
          data. As API readings are taken every 10 minutes and node readings
          every 1 minute, this meant that 9/10 node readings were discarded. The
          final dataset was roughly 1,400 rows. The data used for training
          spanned the period 15 - 27 August.
          \textcolor{red}{Provide dataset as supplementary material.}
          \textcolor{blue}{Made note to add when submitting}
    \item \textbf{Feature set and target data defined:} The feature set from the
          weather API and targets from the node data were defined, and
          unnecessary columns discarded. The database timestamp field was
          transformed into sine and cosine representations of day and year. This
          is necessary when training on a time-series data set as the algorithm
          must be able to understand the cyclical nature of time. For example,
          using raw timestamps would incorrectly suggest to the algorithm that
          the times of 23:00 on day 1 and 00:00 on day 2 are not closely
          related.
    \item \textbf{Dataset split into training data (80\%) and validation data
          (20\%):} The data are split by time so the training data consists of
          the first 80\% of the rows and the validation data the last 20\%.
          These data are then supplied to the model.
    \item \textbf{Iterative training:} For each iteration, the model looks at
          the inputs (training features) and the correct answers (training
          target) of the training rows, and determines where it is getting
          incorrect outputs. It builds a small decision tree that specifically
          aims to correct those mistakes on the training rows and adds that tree
          into itself so its predictions change a little. It then applies the
          updated model to the validation inputs (validation features) and
          compares those predictions to the validation answers (validation
          target) —to see how well the model would do on new "unseen" data. The
          validation data are never used to build the tree; they are only used
          to check the accuracy of the model. If the validation check shows no
          improvement after a number of iterations, the training stops and the
          model keeps the version that performed best on validation.  The
          process will perform a minimum of 50 iterations. I set the maximum
          number of iterations to 250 to prevent the models getting too large,
          as each iteration increases the model size substantially (The humidity
          model is over 40,000 lines long in JavaScript format for example).
\end{enumerate}

Once the ten models had been trained, they were uploaded in JavaScript format to
the web server. An automated function in the webapp provides the models with
data from general forecast hourly up to 48 hours in advance. This general
forecast data was then fed through each of the models to provide a 48 hour
prediction of each sensor variable (temperature, humidity etc.). These
predictions could then be requested by the front end in JSON format to display a
line graph of predicted values for the next 48 hours (see dotted line in Figure
\ref{temperature}).

\section{Results}\label{RES}

The system was evaluated in two key areas. First, we evaluated the performance
of the machine learning forecasting. Then we evaluated the custom hardware by
quantitatively comparing it with commercial alternatives.

The results for machine learning are limited by a small training data set, with
only two weeks of sensor data captured before training. This limitation was
compounded by the fact that the chosen period had virtually no precipitation.
This factor limited the accuracy of the ML model during the evaluation period,
as there was significant rainfall at that time. Therefore, predictions of
humidity and soil moisture were generally less accurate than expected.

\subsection{Machine learning performance}

The accuracy of the machine learning prediction was evaluated over a period of
48 hours from 12am on 2025/08/30 to 12am on 2025/09/01. This was the period
following the training of the model so was not included in the training data.

The performance of the model was compared to two alternatives:
\begin{enumerate}
    \item \textbf{Raw general forecast from OpenWeather:} This was the input to
    the machine learning model and would give an indication of whether the
    accuracy could be improved.
    \item \textbf{Alternative mean average adjustment:} This simple alternative
    model applied a constant mean average adjustment to the above raw forecast.
    This would help identify if a simpler model than the LightGBM process could
    be used instead.
\end{enumerate}

\begin{figure*}[htbp]
\centerline{\includegraphics[width=0.8\textwidth]{figures/results.png}}
\caption{Heatmap results showing relative MAE and RMSE averaged over both nodes}
\label{results}
\end{figure*}

Figure \ref{results} shows the machine learning model performance had mixed
results:

\subsubsection{Wind and gust speed}

One relative success of the model was in the prediction of wind speed. The
general forecast and alternative model were completely unusable in this regard
with MAE of around 200 to 400\%. This overestimate is likely due to the
macro-scale forecast using wind speed data from much higher positions (often
10m) where wind speeds are much higher. Even the adjusted model does not correct
this enough to come close to the true figure with an MAE of 232\%. The machine
learning model achieved a dramatically more accurate prediction with an MAE of
44\%. While still a notable error, this represents a large improvement over the
general forecast, which helps to validate this localised forecasting approach.

\subsubsection{Temperature and humidity}

For temperature and humidity, the ML model did not perform as well. The
performance differences for temperature were minor overall, with all models
achieving a low MAE. The alternative prediction achieved the lowest error (1.3\%
MAE), followed by the ML prediction (2.9\% MAE) and general forecast at (3.4\%).
This suggests that temperature did not differ significantly in the deployed
microclimate compared to the macro-climate, and a simple static bias correction
is quite effective for this prediction.

The humidity forecast however showed a distinctly worse result for machine
learning (14.9\% MAE) compared to both the general forecast (5.5\% MAE) and the
alternative prediction (11.1\%). However in RMSE the three predictions are more
tightly grouped; RMSE is more sensitive to large outliers suggesting the general
forecast had more occasional but large errors.

The poorer humidity performance is an indicator that the model overfit to the
small training sample. With no precipitation in training the model could not
accurately predict how rain would affect the local humidity.


\subsubsection{Soil moisture}

A key benefit of the ML model approach is that soil moisture can be predicted,
which is not usually reported in general weather forecasts. Therefore the
machine learning model is the only model with any prediction for this variable.

The high error for soil moisture (162\% MAE) is a predictable consequence of the
lack of precipitation in the training data. When it rained before and during the
evaluation period the model had not learned the relationship between general
forecasted rain and an increase in soil moisture. This meant the model
incorrectly predicted dry soil for the entire evaluation period. 

\subsection{Hardware evaluation}

The other key focus of this work was to create a prototype hardware network that
was both low-cost and robust enough for the farm environment.

\subsubsection{Range}

The LoRa modules were tested in an urban park in Bristol. While relatively flat,
the park topography still resulted in broken line of sight between the LoRa
modules at roughly the 1,000m mark so communication was not expected beyond this
point. 

\begin{figure}[htbp]
\centerline{\includegraphics[width=0.5\textwidth]{figures/range-test-markers.jpg}}
\caption{Annotated satellite image of range test location with 200m markers}
\label{map}
\end{figure}

\begin{figure}[htbp]
\centerline{\includegraphics[width=0.5\textwidth]{figures/range-test-elevation-profile.jpg}}
\caption{Elevation of test area showing highest point at 1,000m}
\label{elevation}
\end{figure}

Despite the lack of line-of-sight, a final range of 1,200m was achieved. With
the inclusion of the repeater, this gives the system a minimum effective range
of 2,400m. In a real agricultural deployment, the repeater could be placed on a
hill, overcoming line-of-sight issues and enabling a range likely multiples
greater.

\subsubsection{Cost}

The final cost of the hardware was £520. This makes the system more cost
effective than commercial alternatives; many professional grade weather stations
are over 10 times this price which makes them inaccessible for many smaller
farms. 

Table \ref{tab:commercial-comparison} presents a range of similar stations on
the market. Our system is £200 cheaper than the nearest competitor (SenseCAP
S2120) which also uses LoRa for transmission. \textcolor{blue}{Due to the use of
a repeater the estimated range would also double compared to the other LoRa
stations, giving much better range. Additionally, as more sensor nodes are added
this cost advantage becomes even greater, meaning a larger deployment with many
nodes would see even greater savings. The use of LoRa also results in no ongoing
mobile sim card fees as in the case with the HOBO system.}

\textcolor{blue}{However, it should be mentioned that while cheaper there are
some specific areas where the alternative models have an advantage. The most
striking is the relative lack of sensors on the Agriscanner network, only four
sensors vs around 7-11 in other systems. Additionally, the use of hobbyist grade
components means precision is lower in Agriscanner; for example the DHT11 sensor
used for temperature measurements in this prototype is rated for accuracy of
$\pm2.0^{\circ}\mathrm{C}$, whereas the Decentlab system is
$\pm0.6^{\circ}\mathrm{C}$. There are also intangible benefits such as long-term
support and warranties.}

\begin{table*}[t]
  \centering
  \small
  \renewcommand{\arraystretch}{1.2}
  \begin{tabularx}{\textwidth}{l >{\raggedright\arraybackslash}X
      >{\raggedright\arraybackslash}X >{\raggedright\arraybackslash}X
      >{\raggedright\arraybackslash}X >{\raggedright\arraybackslash}X}
    \hline
                                                    & \textbf{Agriscanner
    Network} & \textbf{SenseCAP S2120\cite{pihut:sensecap-s2120-2025}} &
    \textbf{Decentlab Eleven Parameter\cite{alliot:decentlab-eleven-2025}}   &
    \textbf{HOBO weather station kit\cite{weathershop:hobo-rx3000-2025}} &
    \textbf{SparkFun Arduino weather kit\cite{pihut:sparkfun-2025}} \\

    \hline
    Number of sensors                               & 4 & 8 &
    11\textsuperscript{*} & 6 & 7 \\
    Sensor accuracy                                 & Hobbyist & Hobbyist &
                                                    Professional & Professional
                                                    & Hobbyist \\
    Communication type                              & LoRa & LoRa & LoRa &
                                                    Mobile network        & WiFi
                                                    \\
    Update frequency                                & 1 minute & 1 hour & 10
                                                    minutes & 1 hour & 1 minute
                                                    \\
    Readings per hour                               & 60 & 1 & 6 & 10 & 60 \\
    Power source included                           & Yes & Yes & Yes & Yes & No
    \\
    Power source                                    & Solar & Solar & Solar &
    Solar & -- \\
    Batteries recharge?                             & Yes & No & No & Yes & --
    \\
    Reported battery life                           & replace $\sim$ 3 years &
    154 days & Several months & replace 3--5 years & -- \\
    Reported range                                  & -- & 2--10\,km & 2--10\,km
    & Anywhere with 4G & -- \\
    Estimated range                                 & 2.4--20\,km & 1.2--10\,km
                                                    & 1.2--10\,km & -- &
                                                    10--50\,m \\
    IP rating                                       & $\sim$ IP65 & IPX6 & IP66
    & IP66 & None     \\
    Ongoing payment?                                & -- & -- & -- & Yes --
    mobile plan & -- \\
    Ongoing costs p.a                               & \pounds{}0 & \pounds{}0 &
                                                    \pounds{}0 & \pounds{}132 &
                                                    \pounds{}0 \\
    Cost per sensor node                            & \pounds{}177 &
    \pounds{}287 & \pounds{}3{,}272 & \pounds{}4{,}138 & \pounds{}130 \\
    Cost per repeater                               & \pounds{}93 & \pounds{}0 &
                                                    \pounds{}0 & \pounds{}0 &
                                                    \pounds{}0 \\
    Cost per gateway\textsuperscript{**}            & \pounds{}66 & \pounds{}122
                                                    & \pounds{}122 & \pounds{}0
                                                    & \pounds{}0 \\
    Battery cost p.a.          & \pounds{}7 & \pounds{}10 & \pounds{}30 &
                                                    \pounds{}20 & \pounds{}0 \\
    \textbf{Total cost\textsuperscript{***}}       & \textbf{\pounds{}520} &
    \textbf{\pounds{}706} & \textbf{\pounds{}6{,}696} &
    \textbf{\pounds{}8{,}418} & \textbf{\pounds{}260} \\
    \hline
  \end{tabularx}

  \vspace{0.25em}
  \textit{\footnotesize * Sensors missing from Agriscanner: Solar radiation,
    rainfall, barometric pressure, vapor pressure, dew point, wind direction,
    tilt sensor, lightning strike count / distance} \\
  \textit{\footnotesize ** For SenseCAP and Decentlab models the lowest cost
    gateway available is sensecap m2 at £122} \\
  \textit{\footnotesize *** Includes sensors, repeater, gateway, and estimated first-year battery + ongoing costs.}
  \caption{Comparison of weather-station options.}
  \label{tab:commercial-comparison}
\end{table*}

\textcolor{blue}{\subsection{Software evaluation}
The usability of the webapp was also evaluated using a System Usability Scale
\cite{brookeSUS1995}. 15 participants were tasked with performing common basic
tasks on the webapp, such as reading the temperature on a particular date. Then
participants filled out a Likert scale questionnaire of 10 SUS questions.}

\textcolor{blue}{The application achieved a mean SUS score of 87.7, which
exceeds the common benchmark of 68 and indicating a high degree of usability.
Users also had a successful task completion rate of 92\%, which further supports
this conclusion. Furthermore, statistical analysis using a Mann-Whitney U test
revealed no significant difference in satisfaction between mobile and desktop
users ($p > 0.05$), suggesting the mobile design was just as usable as the
desktop interface. However, as the sample of subjects was small these results
should only be taken as an indication of the webapp's usability and not
conclusive evidence.}

\section{Discussion}\label{sec:discussion}
\textcolor{red}{issues to discuss?: e.g, The 'cold start' challange?  Deployment of many micro-stations? Relevance of the forecast (cost/benefit)? etc. }

\textcolor{blue}{\subsection{Current issues and future development cycles}
In addition to the mentioned lack of data for machine learning purposes, another
problem identified was the restart behaviour of the nodes when power was cut out
and then restored. If this happened the device should have entered into
CircuitPython's "safe mode" which was programmed to restart data collection
after a black out or brown out (volatage dips). However the device did not seem
to reliably restart as expected and would often need a manual reset if batteries
drained - which became more frequent in the Autumn months.} \textcolor{blue}{In
future development cycles, the software on the devices will be re-written in
another Python distribution called MicroPython to see if this issue is
resolved.}
 
\textcolor{blue}{\subsection{Relevance of agricultural forecasting}
The study confirms that general macro-scale forecasts often fail to capture
critical local variations. For instance, the general forecast overestimated
local wind speeds with a relative error of over 300\%, likely due to measuring
at higher altitudes. In contrast, the locally trained LightGBM model reduced
this error to 43\%, providing data that is more relevant for precision tasks
such as pesticide spraying.}

\textcolor{blue}{Improved forecasting accuracy would have clear financial value
for farmers. Beyond wind speed, accurate local temperature data can help predict
overnight frost or pest incubation periods (see \cite{haider2017}). Because the
sensor nodes are lower cost than alternatives (\pounds{}177), a farmer could
afford to install more of them across a field to capture climate variation .
This gives a much more useful picture of the farm's weather compared to a single
expensive commercial station, ultimately helping to save money by optimising
when to spray crops or protect them from frost.}

\section{Conclusion}\label{CONC}

This paper presented a methodology for forecasting local microclimates in their
farms integrated into an end-to-end IoT weather station system. This was
designed to provide farmers with the a super-local forecast that macro-scale
"general" forecasts are unable to capture. 

Our evaluation confirmed that our hardware is a viable cost-effective solution
offering superior range to commercial alternatives. More significantly we
trained a LightGBM model to predict the error between general forecasts and
local sensor data to create a localised forecast. This was most effective in the
prediction of wind speed where the model reduced the relative MAE from 333\% to
43\%. Performance in humidity and temperature was not clearly better than the
general forecast but this is almost certainly down to a lack of training data in
the models - particularly the lack of precipitation.

Future work will focus on generating a larger multi-season dataset with the aim
of building a more accurate model. 

%TC:ignore

\bibliography{references}

%TC:endignore
\end{document}
